\documentclass{report}
\usepackage[french]{babel}
\usepackage{graphicx} % Required for inserting images
\usepackage[backend=biber, style=enc]{biblatex}

\title{Computational Humanities in Napoli}
\author{Thibault Clérice}
\date{September 2024}

\addbibresource{bibliography/introduction-to-ch.bib}
\addbibresource{bibliography/cv.bib}
\addbibresource{bibliography/nlp.bib}


\begin{document}

\maketitle

\chapter{Introduction to Computational Humanities}

\section{From digitized humanities to computational humanities through the classics mirror}

\subsection*{Bibliography}

\subsection{The Digital \textit{Incunabula} Period}

\cite{brunner_classics_1993}

\cite{helgerson_cd-rom_1988}

\cite{raben_humanities_1991}

\cite{mylonas_perseus_1993}

\subsection{Conflicts in Rome}

\cite{tombeur_informatique_1997}

\cite{tombeur_pld_1993}

\cite{tombeur_reponse_1993}

\cite{chadwick-healey_droit_1993}

\subsection{To Corpus or Not to Corpus ? From DH to CH ?}

\cite{camps_ou_2018}

\cite{poole_now_2013}

\cite{mcgillivray_tools_2013}


\section{Stylometry and authorship identification}

\paragraph{Invited speaker: Florian Cafiero}

\subsection*{Bibliography}

\section{Ecology and Cultural Heritage Artefacts Survival}

\paragraph{Invited speaker: Mike Kestemont}

\subsection*{Bibliography}

\section{Questioning the Canon}

\subsection*{Bibliography}

\section{Computational Narrative Analysis}

\subsection*{Bibliography}

\cite{haussler2023operationalizing}

\chapter{Natural Language Processing for the Computational Humanities}

This course is split into three lectures and three hands-on session. The hands-on session will target the production of datasets, for which the registered students will spend time annotating in and outside of the class time.

\section{Lectures}

\subsection{Semantics in Natural Language Processing}

\subsection*{Bibliography}

\subsection{Grammatical annotations and models}

\subsection*{Bibliography}

\subsection{Machine Translation}

\paragraph{Invited Speaker: Benoit Sagot}

\subsection*{Bibliography}

\section{Project}

\subsection*{Bibliography}

\chapter{Computer Vision for the Computational Humanities}

This course is split into three lectures and three hands-on session. The hands-on session will target the production of datasets, for which the registered students will spend time annotating in and outside of the class time.

\section{Lectures}

\subsection{Handwritten Text Recognition and the computational humanities}

\subsection*{Bibliography}

\subsection{Distant viewing}

\subsection*{Bibliography}

\cite{arnodchr2022}

\cite{lang2023}

\subsection{Digital Palaeography}

\paragraph{Invited Speaker: Malamatenia Vlachou-Esthatiou}

\subsection*{Bibliography}

\section{Project}

\end{document}
